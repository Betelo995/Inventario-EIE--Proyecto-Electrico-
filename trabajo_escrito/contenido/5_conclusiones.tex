% ----------------------------------------
  \chapter{Conclusiones y recomendaciones}
% ----------------------------------------
\label{C:conclusiones}
\section{Conclusiones}
\begin{itemize}
    \item Se logró crear, entrenar y utilizar una red neuronal capaz de procesar texto por medio de OCR para el procesamiento de las placas de la EIE.
    \item Se creó correctamente una base de datos relacional, que posee una tabla con los campos necesarios para registrar los datos más relevantes del inventariado solicitado por la Universidad de Costa Rica a cada unidad académica.
    \item Se implementó una aplicación con interfaz gráfica intuitiva, que permite a los usuarios registrar activos de forma sencilla y gráfica.
    \item Se sentó una base para la compilación de la aplicación en plataformas Android que toma en cuenta las diferencias para el entorno dentro del diseño.
    \item Se comprobó el funcionamiento integral de la aplicación y su funcionamiento óptimo para diferentes escenarios y posibles errores humanos en el uso de la misma.

    
\end{itemize}


\section{Recomendaciones}
\begin{itemize}
    \item Utilizar contenido educativo de personas que hayan realizado alguna función necesaria previamente como forma de aclaración de la documentación oficial
    \item Delimitar los proyectos de forma más especifica para el tiempo dado durante el curso de Proyecto Eléctrico.
    \item Diseñar los programas por medio de bloques que finalmente funcionen en conjunto, para facilitar el proceso de corrección de errores y de un único diseño extenso.
    \item Utilizar un sistema de control de versiones (cómo git) para poder monitorear el proceso y regresar a versiones previamente desarrolladas que pudisen contener código necesario para una versión más reciente.
\end{itemize}