% LA NOMENCLATURA
% ---------------

% La nomenclatura se realiza con el paquete 'nomencl'. Para ingresar un nuevo elemento, se debe usar el comando \nomenclature{símbolo}{definición}, ya sea en este archivo nomenclatura.tex (más fácil para encontrar y editar), o en cualquier parte del documento (probablemente cuando se introduce una nueva variable o constante). Para más opciones del paquete, favor referirse a su documentación (https://www.ctan.org/pkg/nomencl). También hay una buena guía de uso en https://www.sharelatex.com/learn/Nomenclatures.

% Formato recomendado
% -------------------

% Variable o constante matemática
% \nomenclature{$V$}{Tensión eléctrica}

% Acrónimo
% \nomenclature{TBH}{Para ser honesto (del inglés \textit{To Be Honest})}

% Si únicamente existen acrónimos del inglés, se puede omitir la frase 'del inglés'. La definición no tiene punto al final.

\nomenclature{OCR}{Reconocimiento óptico de caracteres (del inglés \textit{Optical Character Recognition })}
\nomenclature{IEEE}{Instituto de Ingenieros Eléctricos y Electrónicos (del inglés \textit{Institute of Electrical and Electronics Engineers})}
\nomenclature{EIE}{Escuela de Ingeniería Eléctrica de la Universidad de Costa Rica}
\nomenclature{SGBD}{Sistemas de gestión de bases de datos.}
\nomenclature{API}{Interfaz de programación de aplicaciones (del inglés \textit{Application Programming Interface}) }
\nomenclature{BD}{Base de datos}


\printnomenclature